\chapter[label=shared_heap]{Memory management in CHERIoT RTOS}

It is common for embedded systems to avoid heap allocation entirely and preallocate all memory that they will need.
This means that the total amount of memory that a system requires is the sum of the peak memory usage of all components.

The CHERIoT platform is designed to enable safe reuse of memory.
The shared heap allows memory to be dynamically allocated for individual uses and then reused.
This means that the total memory requirement for a system becomes the peak combined usage of all components.
If two components use a lot of memory at different times, they can safely share the same memory.

This chapter covers how to allocate and manage memory.
CHERIoT provides spatial and memory safety so there is a complementary aspect to memory allocation: What do you do when you manage memory incorrectly and CHERIoT catches the error?
Error handling from CHERI exceptions is covered in detail in \ref{handling_errors}.

\section{Understanding allocation capabilities}

The memory allocator uses a capability model.
Every caller of a memory allocation or deallocation function must present a capability that authorises allocation.
This is a \keyword{sealed capability} to an \c{AllocatorCapabilityState} structure.
Sealed capabilities were introduced in \ref{sealing_intro}.

This uses the static sealing mechanism described in \ref{software_capabilities}.
There is no limit to the number of allocator capabilities that a compartment can hold.
Each allocation capability holds an independent quota.

There is no requirement that the sum of all allocation quotas is less than the total available heap space.
You can over-commit memory if you know that it will not all be needed at the same time.
The quota mechanism gives you a way of limiting the total memory consumption of individual compartments (or groups of compartments) and of cleaning up after failure.

\section[label=custom_alloc_capabilities]{Creating custom allocation capabilities}

A compartment may hold different allocation capabilities for different purposes.
The \c{heap_free_all} function allows you to free all memory allocated with a specified capability and so using multiple allocation quotas can be useful for error recovery.

You can forward-declare an allocator capability with the \c{DECLARE_ALLOCATOR_CAPABILITY} macro.
This takes a single argument: the name of the allocator capability.
You can then define the allocator capability with the \c{DEFINE_ALLOCATOR_CAPABILITY} macro, which takes the name and the quota size as arguments.
These can be combined with the \c{DECLARE_AND_DEFINE_ALLOCATOR_CAPABILITY} macro.

\begin{caution}
The allocator capabilities are exposed as COMDATs in C++.
This allows them to be defined in a header and used in multiple translation units.
C does not expose a similar mechanism, so you must use the separate declare and define macros in C if your compartment has multiple compilation units that wish to share an allocator capability and define the capability in a single compilation unit.
\end{caution}

In future versions of CHERIoT RTOS, allocator capabilities are likely to gain additional restrictions (for example, separating the ability to allocate from the ability to claim).

\section{Recalling the memory safety guarantees}

Every pointer to a new allocation provided by the memory allocator is derived from a capability to a large heap region and bounded.
The \keyword{capability monotonicity} guarantees in a CHERI system ensure that a caller cannot expand the bounds of the returned pointer.

The CHERIoT platform provides two additional features for temporal safety.
These both depend on a revocation bitmap, a shadow memory space that stores one bit per eight bytes of heap memory.
When an object is freed, the allocator paints the bits associated with it.

The \keyword{load filter} is part of a CHERIoT core.
When a capability is loaded from memory into a register, the load filter checks the revocation bit associated with the base of the capability and clears the tag bit if the capability points to freed memory (filtering out dangling pointers).
The load filter ensures that you cannot load, and therefore cannot try to use, any dangling pointers.
This gives deterministic use-after-free protection; any attempt to use a pointer to a deallocated object will trap.
The object is then placed in \keyword{quarantine}.

The \keyword{revoker} periodically scans all memory and invalidates any pointers whose base address points to a deallocated object.
The monotonicity of bounds ensures that the base of a capability always points either somewhere within the allocation or, if the length is zero, to the word immediately after it.

\begin{note}
The allocator marks the metadata between allocations as freed.
This means that a zero-length capability to the end of an object is likely to be untagged.
\end{note}

The load filter ensures that no new pointers to deallocated objects can appear in memory and so the revocation sweep can proceed asynchronously.
Any object that is in quarantine at the start of a sweep is safe to remove from quarantine at the end.

This combination of features allows the allocator to provide complete spatial and temporal safety for heap objects.

\section{Allocating with an explicit capability}

\functiondoc{heap_allocate}
\amanda{Within this functiondoc, in the fifth paragraph, the word "fulfiled" should be "fulfilled"}
\amanda{Within this functiondoc, in the seventh paragraph, asterisks are used for emphasis instead of textem.}

\functiondoc{heap_free}

The \c{heap_allocate} and \c{heap_free} functions take a capability, as described above, that authorises allocation and deallocation.
When an object is allocated with an explicit capability, it may be freed only by presenting the same capability.
This means that, if you pass a heap-allocated buffer to another compartment, that compartment cannot free it unless you also pass the authorising capability.

\begin{note}
	The allocation uses a timeout because the allocation API is able to block if insufficient memory is available.
	In contrast the deallocation API will always make progress.
	The allocator uses a priority-inheriting lock, which is dropped while blocking.
	If a high-priority thread frees memory while a lower-priority thread owns the lock then the lower-priority thread will wake up, complete its allocation or deallocation, release the lock, and allow the higher-priority thread to resume.
\end{note}

If you need to clean up all memory allocated by a particular capability, \c{heap_free_all} will walk the heap and deallocate everything owned by that capability.
This is useful when a compartment has crashed, to reclaim all of its heap memory.

\functiondoc{heap_free_all}

\section{Using C/C++ default allocators}

If you are porting existing C/C++ code then it is likely that it uses \c{malloc} / \c{free} or the C++ \c{new} / \c{delete} operators.
These are implemented as wrappers around \c{heap_allocate} and \c{heap_free} that pass \c{MALLOC_CAPABILITY} as the authorising capability.
You can also pass this capability explicitly to allocate things from the same quota as the standard allocation routines.

\begin{note}
	\c{MALLOC_CAPABILITY} is a macro referring to the default allocation capability \textem{in the current compartment}.
	It refers to a different capability in every compartment.
\end{note}

You can control the amount of memory provided by this capability by defining the \c{MALLOC_QUOTA} for your compartment.
If a compartment is not supposed to allocate memory on its own behalf, you can define \c{CHERIOT_NO_AMBIENT_MALLOC}.
This will disable C's \c{malloc} and \c{free} and C++'s global \c{new} and \c{delete} operators.
Defining \c{CHERIOT_NO_NEW_DELETE} will disable the global C++ operator \c{new} and \c{delete}, but leave \c{malloc} and \c{free} available.

Defining these does not prevent memory allocation; you can still define non-default allocator capabilities and use them directly, but it prevents accidental allocation.

\section{Defining custom allocation capabilities for \c{malloc} and \c{free}}

If you simply wish to change the quota that is available to \c{malloc} and \c{free} then you can define \c{MALLOC_QUOTA} when compiling your compartment.
If you require more control, such as controlling the compilation unit that contains the definition of the allocator capability, then you can define \c{CHERIOT_CUSTOM_DEFAULT_MALLOC_CAPABILITY}.
This macro will cause \c{stdlib.h} to provide a forward declaration of the default allocator capability, but not to define it.
You must define it as described in \ref{custom_alloc_capabilities}.

This is most useful for C compartments with multiple compilation units.
These will need to define the malloc capability in a single compilation unit.

\begin{note}
This limitation will be removed in a future toolchain iteration.
\end{note}

\section[label=token_apis]{Allocating on behalf of a caller}

Sometimes a compartment needs to be able to allocate memory but that memory is not logically owned by the compartment.
This pattern appears even in the core of the RTOS.
The compartment that provides message queues, for example, allocates memory on behalf of a caller. 
It does not hold the right to allocate memory on its own behalf.
It does this by taking an allocator capability as an argument and forwarding it to the allocator.

Often, if a compartment is allocating on behalf of a caller, it needs to ensure that the caller doesn't tamper with the object.
The token APIs provide a lightweight mechanism for doing this.

\functiondoc{token_sealed_unsealed_alloc}

\functiondoc{token_obj_unseal}

\functiondoc{token_obj_destroy}

When the delegated compartment calls \c{token_sealed_unsealed_alloc}, you must provide two capabilities:

\begin{itemize}
	\item{An allocator capability.}
	\item{A permit-seal sealing capability.}
\end{itemize}

The first of these authorises memory allocation, the second authorises sealing.
The CHERIoT ISA includes only three bits of object type space in the capability encoding and so the allocator provides a virtualised sealing mechanism.
This allocates an object with a small header containing the sealing type and returns a sealed capability to the entire allocation and an unsealed capability to all except the header.

The unsealed capability can be used just like any other pointer to heap memory.
The sealed capability can be used with \c{token_obj_unseal} to retrieve a copy of the unsealed capability.
The \c{token_obj_unseal} function requires a permit-unseal capability whose value matches the permit-seal capability passed to \c{token_sealed_unsealed_alloc}.

\begin{note}
The virtualised sealing mechanism must be able to derive an accurate capability for the object excluding the header.
This is trivial for objects up to a little under 4 KiB.
After that, the allocator will create some padding.
The padding is placed at the \textem{start} of the allocation, so you can see how much is there by querying the base and address of the returned (sealed) capability.
\end{note}

An object allocated in this way can be deallocated only by presenting \textem{both} the allocator capability and the sealing capability that match the original allocation.
This is very convenient for compartments that expose services because the memory cannot go away while they are using it and can be reclaimed only when the same caller (or something acting on its behalf) authorises the deallocation.

\section[label=heap_claim]{Ensuring that heap objects are not deallocated}

If malicious caller passes a compartment a buffer and then frees it, the callee can be induced to trap.
There are some situations where this is acceptable.
In some cases, compartments exist in a hierarchical trust relationship and it's fine for a more-trusted compartment to be able to crash a less-trusted one.
In other cases, the compartment is fault tolerant.
For example, the scheduler ensures that its data structures are in a consistent state before performing any operations on user-provided data that may trap.
As such, it can unwind to the caller and, at worst, leak memory owned by the caller.

In situations involving mutual distrust, the callee needs to \keyword{claim} the memory to prevent its deallocation.
The \c{heap_claim} function allows you to place a claim on an object.
The claim is dropped by calling \c{heap_free}.

While you have a claim on an object, that object counts towards your quota.
You can claim the same object multiple times. 
Each time adds a new claim to the object but (if it is already claimed with that quota) does not consume quota.

\begin{note}
You can pass a capability with bounds that do not cover an entire object to \c{heap_claim} but your claim will cover the entire object because you cannot free part of an object.
\end{note}

\functiondoc{heap_claim}

If you need to ensure that an allocation remains valid for a brief, scoped period then \c{heap_claim_ephemeral} may be more useful.
This function places an ephemeral claim on one or two objects.

\functiondoc{heap_claim_ephemeral}
\amanda{Within this functiondoc, in the fourth paragraph "compartment_helpers" should be formatted as code.}

Every thread has two \keyword{hazard slots} that can hold pointers.
The \c{heap_claim_ephemeral} function manages these two slots.
These are cleared on every cross-compartment call and can be cleared explicitly by passing \c{NULL} to \c{heap_claim_ephemeral}.

If a pointer passed to \c{heap_free} is present in the allocator, the allocator will defer freeing the object.
Writing to the hazard slots is very fast.
Unlike \c{heap_claim}, this does not require a cross-compartment call.

\begin{caution}
	Any claim applied with \c{heap_claim_ephemeral} is lost on \textem{any} cross-compartment call.
	This includes any blocking operation, which will invoke the scheduler.
	In general, do not use \c{heap_claim_ephemeral} for anything other than a local read or write of a single object.
\end{caution}

The \c{heap_claim_ephemeral} API is intended for very brief accesses to objects.
You can claim two pointers to support the common pattern of \c{memcpy} between two caller-provided (i.e. untrusted) buffers.
You can claim both and then copy between them.
