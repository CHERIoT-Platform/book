\chapter[label=debug]{Features for debug builds}

CHERIoT provides a small set of APIs for use in debug builds in `debug.hh`.
These include:

\begin{itemize}
	\item{Rich log messages}
	\item{Assertions with error messages}
	\item{Invariants that are checked in release builds but provide debugging help only in release builds}
\end{itemize}

All of the message-producing aspects of these APIs use direct access to the UART.
This can cause the messages to be interleaved but ensures that they are generated even if part of the system has crashed or deadlocked.

Access to the UART will show up in the linker report.
You should ensure that your auditing checks ensure that you have not left debug access to the UART enabled in release builds.

\section{Enabling per-component debugging}

Debug builds can often be significantly larger than release builds.
They contain more code and potentially large strings for debug messages.
CHERIoT RTOS is designed to allow debugging features to be turned on on a per-compartment basis to help mitigate this.
You can see this in the core components.
If you run \command{xmake config --help} in a firmware build, you will see this at the end of the output:

\begin{console}
--debug-token_library=[y|n] Enable verbose output and assertions in the token_library
--debug-allocator=[y|n]     Enable verbose output and assertions in the allocator
--debug-loader=[y|n]        Enable verbose output and assertions in the loader
--debug-scheduler=[y|n]     Enable verbose output and assertions in the scheduler
\end{console}

Each of the core components allows extra debugging modes to be enabled independently, rather than via a global debug-mode switch.
Adding something similar requires two changes in your \file{xmake.lua} file.
The first line, at top-level scope, declares the option:

\begin{luasnippet}
debugOption("myComponent")
\end{luasnippet}

With this, you will get a message in \command{xmake config --help} like the one above, but it won't actually do anything.
You must also opt your compartment or library into debugging support by adding the corresponding rule in the description of your compartment or library:

\begin{luasnippet}
compartment("myComponent")
	add_rules("cheriot.component-debug")
\end{luasnippet}

By default, this assumes that the \lua{debugOption} that you've provided has the same name as the target.
Sometimes, it's useful to have a single debug option that enables or disables debugging for multiple components.
You can set the \lua{cheriot.debug-name} target property in your component to the name that you expect, with a line like this:

\begin{luasnippet}
compartment("myComponent")
	add_rules("cheriot.component-debug")
	on_load(function (target)
		target:set('cheriot.debug-name', "nameOfDebugOption")
	end)
\end{luasnippet}

Now, your compartment will be compiled with a macro that starts with `DEBUG_` and ends with the name of the debug option in all capitals.
In the first example above, this would be \c{DEBUG_MYCOMPONENT}.

This can then be used with the \cxx{ConditionalDebug} class from \file{debug.hh}.
This is typically used as follows:

\begin{cxxsnippet}
using Debug = ConditionalDebug<DEBUG_MYCOMPONENT, "My component">;
\end{cxxsnippet}

The first template parameter is a boolean value that indicates whether this component is being debugged.
The second is a free-form string literal that will be prepended (in magenta) to any debug line.

The rest of this chapter will assume that the \cxx{Debug} type has been defined in this way.

\section{Generating log messages}

Printing log messages is the simplest use of the debug APIs.
The \cxx{Debug::log()} function takes a format string and then a set of arguments.
This is similar to \c{printf} or \cxx{std::format}, inserting the arguments into the output, replacing placeholders.
The syntax here is modelled on \cxx{std::format}, but does not currently accept any format modifiers.
The \cxx{{\}} syntax for placeholders makes it possible to add modifiers in the future.
This class is designed to avoid needing heap allocator or large amounts of stack space and so is intentionally less flexible than a general-purpose formatting library.

Unsigned integers are printed as hex.
Signed integers are printed as decimal.
Floating point numbers are not supported.
Individual characters are printed as characters, strings (either \c{const char*} or \cxx{std::string_view}) are printed as strings.

Enumerated types are converted to strings using the Magic Enum library and printed with their numeric value in brackets.
This has some limitations (in particular, by default, it does not work with very large enumeration values).
It also requires capability relocations because it generates tables of strings.
If you compile a compartment with \c{CHERIOT_AVOID_CAPRELOCS} defined then enumerations will be printed as numeric values.

Two other types have rich formatted output.
\cxx{PermissionSet} objects (see \ref{cheri_capability_cpp}) are printed using the characters from the tables in \ref{permissions}.
Capabilities (either as raw pointers or instances of the \cxx{CHERI::Capability} class) are printed in full detail.
Printing a capability will give a block that looks something like this:

\begin{console}
0x2004cc8c (v:1 0x2004cc8c-0x2004cc90 l:0x4 o:0x0 p: G RWcgm- -- ---)
\end{console}

This starts with the address and then has the metadata in brackets.
The metadata includes the tag (valid) bit, then the range, then the length, object type, and permissions.

\section{Asserting invariants}

Assertions and invariants use the same formatting infrastructure as lthe log message code.
In debug mode (for this component), the following two are equivalent:

\begin{cxxsnippet}
Debug::Invariant(theAnswer == 42, "The answer was {\}, expected 42", theAnswer);
Debug::Assert(theAnswer == 42, "The answer was {\}, expected 42", theAnswer);
\end{cxxsnippet}

They will both check whether then answer is 42 and, if not, print a message to the UART telling you what the real value was.
They will then issue an invalid instruction.
If your compartment does not have an error handler (see \ref{handling_errors}, then this will unwind to the compartment that called you.
If it does, then you can handle this just like any other error.

In release builds, assertions are removed entirely.
Invariants are still checked, but no longer log a message on failure, they just trigger an illegal instruction.

In some cases, you may find that the expression that calculates the assertion condition is expensive and the compiler does not successfully optimise it away in release builds.
In this case, you can use the version that takes a lambda instead:

\begin{cxxsnippet}
Debug::Assert([]() { return someExpensiveCheck(); \}, "An expensive check failed");
\end{cxxsnippet}

The lambda is never executed in release builds and so the compiler will strip it away.
You can also use this form if you have multiple steps (which may have side effects) leading up to the assertion condition.
