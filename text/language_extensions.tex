\chapter[label=language_extensions]{C/C++ extensions for CHERIoT}

The CHERIoT platform adds a small number of C/C++ annotations to support the compartment model.

\section{Exposing compartment entry points}

Compartments are discussed in detail in \ref{compartments}.
A compartment can expose functions as entry points via a simple attribute.

The \c{cheri_compartment({name\})} attribute specifies the name of the compartment that defines a function.
This is used in concert with the \flag{-cheri-compartment=} compiler flag.
This allows the compiler to know whether a particular function (which may be in another compilation unit) is defined in the same compartment as the current compilation unit, allowing direct calls for functions in the same compilation unit and cross-compartment calls for other cases.

This can be used on either definitions or declarations but is most commonly used on declarations.

If a function is defined while compiling a compilation unit belonging to a different compartment then the compiler will raise an error.
In CHERIoT RTOS, this attribute is always used via the \c{__cheri_compartment({name\})}` macro.
This makes it possible to simply use \c{#define __cheri_compartment(x)} when compiling for other platforms.

Most of the time, you will not need to worry about the compiler flags directly.
The \command{xmake} provided by CHERIoT RTOS will set the compiler flags for you automatically.
\ref{lst:simpleexport} shows the prototype of a trivial function that increments an integer that is private to a compartment.

\codelisting[filename=examples/c_extensions/interface.h,marker=compartment_export,label=lst:simpleexport,caption="Exporting a function for use by other compartments from a header."]{}

The body of this function is then shown in \ref{lst:incrementfn}.
Note that this does not require the attribute, it is inherited from the prototype.
If you forget to include the header, you will see a linker error about a missing symbol.
\fixme{I thought we now had a better linker error but I can't reproduce it.}

\codelisting[filename=examples/c_extensions/compartment.cc,marker=increment,label=lst:incrementfn,caption="The body of a function that is exposed for cross-compartment calls."]{}

The build system specifies the \flag{-cheri-compartment=} flag based on the \lua{compartment} target definition in the \file{xmake.lua}.
\ref{lst:xmakecompartment} shows this for the simple example compartment.

\lualisting[filename=examples/c_extensions/xmake.lua,marker=compartment,label=lst:xmakecompartment,caption="Build system code for defining a compartment."]{}

If you get the compartment name wrong, the compiler will generate an error.
For example, if you change the compartment name in this example to \c{"hello"} in the source code, you will see the following:

\begin{console}
error: entry.cc:15:35: error: CHERI compartment entry declared for compartment 'hello' but implemented in 'entry' (provided with -cheri-compartment=)
   15 | void __cheri_compartment("hello") entry()
      |                                   ^
1 error generated.
\end{console}

\section[label=cheri_callback]{Passing callbacks to other compartments.}

The \c{cheri_ccallback} attribute specifies a function that can be used as an entry point by compartments that are passed a function pointer to it.
This attribute must also be used on the type of function pointers that hold cross-compartment invocations.
Any time the address of such a function is taken, the result will be a sealed capability that can be used to invoke the compartment and call this function.

\begin{note}
	The compiler does not know, when calling a callback, whether it points to the current compartment.
	As such, calling a CHERI callback function will \textem{always} be a cross-compartment call, even if the target is in the current compartment.
\end{note}

This attribute can also be used via the \c{__cheri_callback} macro, which allows it to be defined away when targeting other platforms.

\ref{lst:callbackexport} shows both how to declare a \c{typedef} for a function pointer type that can be used for cross-compartment callbacks and how to expose a function that takes one.
This is a simple function that will increment a private counter and invoke the callback.

\codelisting[filename=examples/c_extensions/interface.h,marker=compartment_export_callback,label=lst:callbackexport,caption="Exposing a function that takes a cross-compartment callback for use by other compartments."]{}

The implementation of this function (\ref{lst:monotoniccallbackfn}) calls it just as it would call any other function pointer.
The difference is dealt with entirely by the compiler.
For a normal call, the compiler will emit a simple jump-and-link to the address, whereas in this case it will invoke the switcher (see \ref{_changing_trust_domain_with_the_switcher}) with the callback as an extra argument.

Every function that's exposed for cross-compartment invocation has an entry in the compartment's \keyword{export table}, containing the metadata that the switcher will use.
Every function that is directly called by another compartment will then have an entry in the calling compartment's \keyword{import table} that the loader will initialise with a sealed capability to the export table entry.
Callback functions work in a similar way, except that the import-table entry is for the compartment that exposes the callback.

When you take the address of a callback function, the compiler simply inserts a load of the import-table entry, giving exactly the same kind of sealed capability that you would use for direct cross-compartment calls.
At the call site, the only difference between a direct cross-compartment call and a callback is that the former will contain the load from the import table, whereas the latter will simply move the callback into the register that is used to pass the callee to the switcher.

\codelisting[filename=examples/c_extensions/compartment.cc,marker=monotonic,label=lst:monotoniccallbackfn,caption="The body of a function that invokes a cross-compartment callback."]{}

The callback is then declared just like any other function, but with the correct attribute, as shown in \ref{lst:declaresimplecallback}.

\begin{note}
	The function attributes can be provided either before the start of the function or before the function name (after the return type).
	In some cases, the latter can avoid ambiguity (the attribute definitely applies to the function, not to the return type), but both are equivalent the rest of the time.
\end{note}

\codelisting[filename=examples/c_extensions/entry.cc,marker=callback,label=lst:declaresimplecallback,caption="A function that can be invoked as a cross-compartment callback."]{}

The callback function is passed just like any other function pointer, as shown in \ref{lst:passsimplecallback}.
Note that the two ways of taking the address of a function in C/C++ (\c{callback} and \c{&callback}) are equivalent.
Both work, some people prefer the former because it is more concise, others prefer the latter because it is a visual marker that a pointer is being constructed.

\codelisting[filename=examples/c_extensions/entry.cc,marker=compartment_call,label=lst:passsimplecallback,caption="A function that can be invoked as a cross-compartment callback."]{}

\section{Exposing library entry points}

Libraries are discussed in \ref{compartments}.
Like compartments, they can export functions, via a simple annotation.

The \c{cheri_libcall} attribute specifies that this function is provided by a library (shared between compartments).
Libraries may not contain any writeable global variables.
This attribute is implicit for all compiler built-in functions, including \c{memcpy} and similar freestanding C environment functions.
As with \c{cheri_compartment()}, this may be used on both definitions and declarations.

This attribute can also be used via the \c{__cheri_libcall} macro, which allows it to be defined away when targeting other platforms.



\section{Interrupt state control}

The \c{cheri_interrupt_state} attribute (commonly used as a C++11 / C23 attribute spelled \c{cheri::interrupt_state}) is applied to functions and takes an argument that is either:

\begin{description}
	\item[tag=enabled]{, to enable interrupts when calling this function.}
	\item[tag=disabled]{, to disable interrupts when calling this function.}
	\item[tag=inherit]{, to not alter the interrupt state when invoking the function.}
\end{description}

For most functions, \c{inherit} is the default.
For cross-compartment calls, \c{enabled} is the default and \c{inherit} is not permitted.

The compiler may not inline functions at call sites that would change the interrupt state and will always call them via a sentry capability set up by the loader.
This makes it possible to statically reason about interrupt state in lexical scopes.

If you need to wrap a few statements to run with interrupts disabled, you can use the convenience helper \cxx{CHERI::with_interrupts_disabled}.
This is annotated with the attribute that disables interrupts and invokes the passed lambda.
This maintains the structured-programming discipline for code running with interrupts disabled: it is coupled to a lexical scope.

\functiondoc{with_interrupts_disabled}

\section{Importing MMIO access}

The \c{MMIO_CAPABILITY({type\}, {name\})} macro is used to access memory-mapped I/O devices.
These are specified in the board definition file by the build system.
The \c{DEVICE_EXISTS({name\})} macro can be used to detect whether the current target provides a device with the specified name.

The \c{type} parameter is the type used to represent the MMIO region.
The macro evaluates to a \c{volatile {type\} *}, so \c{MMIO_CAPABILITY(struct UART, uart)} will provide a \c{volatile struct UART *} pointing (and bounded) to the device that the board definition exposes as \c{uart}.

\section{Manipulating capabilities with C builtins}

The compiler provides a set of built-in functions for manipulating capabilities.
These are typically of the form \c{__builtin_cheri_{noun\}_{verb\}}.
You can read all of the fields of a CHERI capability with `get` as the verb and the following nouns:

\begin{description}
	\item[tag=address]{The current address that's used when the capability is used a pointer.}
	\item[tag=base]{The lowest address that this authorises access to.}
	\item[tag=length]{The distance between the base and the top.}
	\item[tag=perms]{The architectural permissions that this capability holds.}
	\item[tag=sealed]{Is this a sealed capability?}
	\item[tag=tag]{Is this a valid capability?}
	\item[tag=type]{The type of this capability (zero means unsealed).}
\end{description}

The verbs vary because they express the \keyword{guarded manipulation} guarantees for CHERI capabilities.
You can't, for example, arbitrarily set the permissions on a capability, you can only remove permissions.
Capabilities can be modified with the nouns and verbs listed in \ref{tbl:cap_modification_builtins}.

\begin{table} % W: -> unmatched "\begin{table}"
	\begin[cols="16%fw 20%fw 64%fw"]{tabular} % W: Use either `` or '' as an alternative to `"'. (18)
		\tr{ \th{Noun} \th{Modification verb} \th{Operation}} % W: possible unwanted space at "{"
		\tr{\td{\c{address}} \td{\c{set}}             \td{Set the address for the capability.}}
		\tr{\td{\c{bounds}}  \td{\c{set}}             \td{Sets the base at or below the current address and the length at or above the requested length, as closely as possible to give a valid capability}}
		\tr{\td{\c{bounds}}  \td{\c{set_exact}}       \td{Sets the base to the current address and the length to the requested length or returns an untagged capability if the result is not representable.}}
		\tr{\td{\c{perms}}   \td{\c{and}}             \td{Clears all permissions except those provided as the argument.}}
		\tr{\td{\c{tag}}     \td{\c{clear}}           \td{Invalidates the capability but preserves all other fields.}}
	\end{tabular} % W: <- unmatched "\end{tabular}"
	\caption[label=tbl:cap_modification_builtins]{CHERI capability manipulation builtin functions}
\end{table} % W: <- unmatched "\end{table}"


Setting the object type is more complex.
This is done with \c{__builtin_cheri_seal}, which takes an authorising capability (something with the permit-seal permission) as the second argument and sets the object type of the result to the address of the sealing capability.
Conversely, \c{__builtin_cheri_unseal} uses a capability with the permit-unseal capability and address matching the object type to restore the original unsealed value.

\section{Comparing capabilities with C builtins}

By default, the C/C++ `==` operator on capabilities compares only the address.

\begin{note}
	This is subject to change in a future revision of CHERI C.
	It makes porting some existing code easier, but breaks the substitution principle (if \c{a == b}, you would expect to be able to use \c{b} or \c{a} interchangeably).
\end{note}

You can compare capabilities for exact equality with \c{__builtin_cheri_equal_exact}.
This returns true if the two capabilities that are passed to it are identical, false otherwise.
Exact equality means that the address, bounds, permissions, object type, and tag are all identical.
It is, effectively, a bitwise comparison of all of the bits in the two capabilities, including the tag bits.

Ordered comparison, using operators such as less-than or greater-than, always operate with the address.
There is no total ordering over capabilities.
Two capabilities with different bounds or different permissions but the same address will return false when compared with either `<` or `>`.

This is fine according to a strict representation of the C abstract machine because comparing two pointers to different objects is undefined behaviour.
It can be confusing but, unfortunately, there is no good alternative.
Comparison of pointers is commonly used for keying in collections.
For example, the C++ \cxx{std::map} class uses the ordered comparison operators for building a tree and relies on it working correctly for keys that are pointers.
Ideally, these would explicitly operate over the address, but that would require invasive modifications when porting to CHERI platforms.

In general, in new code, you should avoid comparing pointers for anything other than exact equality, unless you are certain that they have the same base and bounds.
Instead, be explicit about exactly what you are testing.
Do you care if the permissions are different?
Do you care about the bounds?
Do you care if the value is tagged?
Or do you just want to care about the address?
In each case, you should explicitly compare the components of the capability that you care about.

You can also compare capabilities for subset relationships with \c{__builtin_cheri_subset_test}.
This returns true if the second argument is a subset of the first.
A capability is a subset of another if every right that it conveys is held by the other.
This means the bounds of the subset capability must be smaller than or equal to the superset and all permissions held by the subset must be held by the superset.

\section{Sizing allocations}

CHERI capabilities cannot represent arbitrary bases and bounds.
The larger the bounds, the more strongly aligned the base and bounds must be.

NOTE: The current CHERIoT encoding gives byte-granularity bounds for objects up to 511 bytes, then requires one more bit of alignment for each bit needed to represent the size, up to 8 MiB.
Capabilities larger than 8 MiB cover the entire address space.
This is ample for small embedded systems where most compartments or heap objects are expected to be under tens of KiBs.
Other CHERI systems make different trade offs.

Calculating the length can be non-trivial and can vary across CHERI systems.
The compiler provides two builtins that help.

The first, \c{__builtin_cheri_round_representable_length}, returns the smallest length that is larger than (or equal to) the requested length and can be accurately represented.
The compressed bounds encoding requires both the top and base to be aligned on the same amount and so there's a corresponding mask that needs to be used for alignment.
The \c{__builtin_cheri_representable_alignment_mask} builtin returns the mask that can be applied to the base and top addresses to align them.

\section[label=cheri_capability_cpp]{Manipulating capabilities with \cxx{CHERI::Capability}}

The raw C builtins can be somewhat verbose.
CHERIoT RTOS provides a \cxx{CHERI::Capability} class in \file{cheri.hh} to simplify inspecting and manipulating CHERI capabilities.

These provide methods that are modelled to allow you to pretend that they give direct access to the fields of the capability.
For example, you can write:

\begin{cxxsnippet}
capability.address() += 4;
capability.permissions() &= permissionSet;
\end{cxxsnippet}

This modifies the address of \cxx{capability}, increasing it by four, and removes all permissions not present in \cxx{permissionSet}.
Other operations are also defined to be orthogonal.

Permissions are exposed as a \cxx{PermissionSet} object.
This is a \cxx{constexpr} class that provides a rich set of operations on permissions.
This can be used as a template parameter and can be used in static assertions for compile-time validation of derivation chains.
The loader makes extensive use of this class to ensure correctness.

\begin{caution}
	The equality comparison for \cxx{CHERI::Capability} uses exact comparison, unlike raw C/C++ pointer comparison.
	This is less confusing for new code (it respects the substitution principle) but users may be confused that \c{a == b} is true but \cxx{Capability{a\} == Capability{b\}} is false.
\end{caution}

See \file{cheri.hh} for more details and for other convenience wrappers around the compiler builtins.

