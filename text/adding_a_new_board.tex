\chapter[label=new_board]{Adding a new board}

CHERIoT RTOS uses a JSON file to describe the target.
At a first glance, this looks similar to a flattened device tree (FDT) source file.
Both contain a layout of memory and locations of devices but the CHERIoT RTOS board description file also contains a lot of information that is useful only at build time, such as the locations of header files and preprocessor defines for the platform.

When you want to create a board support package (BSP) for a new CHERIoT configuration, this is the first place to start.
The CHERIoT RTOS build system allows board description files to be specified either as names in the \file{sdk/boards} directory or as file paths.
Anything that has been contributed to the CHERIoT RTOS repository will use the former mechanism, anything distributed separately will use the latter.

\section{Specifying memory layout}

CHERIoT RTOS has to differentiate three kinds of memory in the configuration

\begin{itemize}
	\item{Code and read-only global data, which cannot contain pointers to revokable heap memory.}
	\item{Globals and stacks, which may contain pointers to revokable heap memory.}
	\item{The heap, which may contain pointers to revokable heap memory and may itself be revoked.}
\end{itemize}

The allocator is given a capability to the revocation bitmap covering the last range.
The revoker must be configured to include (at least) the latter two in its scans.
The first category is safe to ignore.

\begin{note}
A future version of CHERIoT RTOS may differentiate between code (and non-capability read-only data) and read-only data so that the former can be run from memory that does not support tags and the latter from tag-carrying memory.
\end{note}

The memory layout will put code then globals into memory and then the heap afterwards.
In most systems, there is more code than heap and so, to reduce costs, not all memory needs to support tags.

Our security guarantees for the shared heap depend on the mechanism that allows the allocator to mark memory as quarantined.
Any pointer to memory in this region is subject to a check (by the hardware) on load: if it points to deallocated memory then it will be invalidated on load.
This mechanism is necessary only for memory that can be reused by different trust domains during a single boot.
Memory used to hold globals and code does not require it and so an implementation may save some hardware and power costs by supporting these temporal safety features for only a subset of memory.
As such, we require a range of memory that is used for static code and data ('instruction memory') that is not required to support this mechanism and an additional range that *must* support this for use as the shared heap ('heap memory').
Implementations may choose not to make this separation and provide a single memory region.
At some point, we expect to further separate the mutable and immutable portions of instruction memory so that we can support execute in place.

Instruction memory is described by the \json{instruction_memory} property.
This must be an object with a \json{start} and \json{end} property, each of which is an address.

The region available for the heap is described in the \json{heap} property.
This must describe the region over which the load filter is defined.
If its \json{start} property is omitted, then it is assumed to start in the same place as instruction memory.

The Sail board description has a simple layout:

\begin{jsonsnippet}
    "instruction_memory": {
        "start": 0x80000000,
        "end": 0x80040000
    \},
    "heap": {
        "end": 0x80040000
    \},
\end{jsonsnippet}

This starts instruction memory at the default RISC-V memory address and has a single 256 KiB region that is used for both kinds of memory.

\section{Exposing MMIO Devices}

Each memory-mapped I/O device is listed as an object within the \json{devices} field.
The name of the field is the name of the device and must be an object that contains a \json{start} and either a \json{length} or \json{end} property that, between them, describe the memory range for the device.
Software can then use the \c{MMIO_CAPABILITY} macro with the name of the device to get a capability to that device's MMIO range and can use \c{#if DEVICE_EXISTS(device_name)} to conditionally compile code if that device exists.

The Sail model is very simple and so provides only three devices:

\begin{jsonsnippet}
    "devices": {
        "clint": {
            "start": 0x2000000,
            "length": 0x10000
        \},
        "uart": {
            "start": 0x10000000,
            "end":   0x10000100
        \},
        "shadow" : {
            "start" : 0x83000000,
            "end"   : 0x83001000
        \}
    \},
\end{jsonsnippet}

This describes the core-local interrupt controller (\json{clint}), a UART, and the shadow memory used for the temporal safety mechanism (\json{shadow}).
The UART, for example, is referred to in source using \c{MMIO_CAPABILITY(struct Uart, uart)}, which evaluates to a \c{volatile struct Uart *}, giving a capability to this device.

\section{Defining interrupts}

External interrupts should be defined in an array in the \json{interrupts} property.
Each element has a \json{name}, a \json{number} and a \json{priority}.
The name is used to refer to this in software and must be a valid C identifier.
The number is the interrupt number.
The priority is the priority with which this interrupt will be configured in the interrupt controller.

Interrupts may optionally have an \json{edge_triggered} property (if this is omitted, it is assumed to be false).
If this exists and is set to true then the interrupt is assumed to fire when a condition first holds, rather than to remain raised as long as a condition holds.
Interrupts that are edge triggered are automatically completed by the scheduler; they do not require a call to \c{interrupt_complete}.

\section{Controlling hardware features}

Some properties define base parts of hardware support.
The \json{revoker} property is either absent (no temporal safety support), \json{"software"} (revocation is implemented via a software sweep) or \json{"hardware"} (there is a hardware revoker).
We expect this to be \json{"hardware"} on all real implementations, the software revoker exists primarily for the Sail model and the no temporal safety mode only for benchmarking the overhead of revocation.

If the \json{stack_high_water_mark} property is set to true, then we assume the CPU provides CSRs for tracking stack usage.
This property is primarily present for benchmarking as all of our targets currently implement this feature.

\section{Specifying clock speeds}

The clock rate is configured by two properties.
The \json{timer_hz} field is the number of timer increments per second, typically the clock speed of the chip (the RISC-V timer is defined in terms of cycles).
The \json{tickrate_hz} specifies how many scheduler ticks should happen per second.
\fixme{Link to tick descriptions}

\section{Supporting conditional compilation}

The \json{defines} property specifies any pre-defined macros that should be set when building for this board.
The \json{driver_includes} property contains an array (in priority order) of include directories that should be added for this target.
Each of the paths in \file{driver_includes} is, by default, relative to the location of the board file (which allows the board file and drivers to be distributed together).
Optionally, it may include the string \json{$(sdk)}, which will be replaced by the full path of the SDK directory.
For example, \json{"$(sdk)/include/platform/generic-riscv"} will expand to the generic RISC-V directory in the SDK.

The driver headers use \c{#include_next} to include more generic files and so it is important to list the directories containing your overrides first.

\section{Simulation support}

There are two properties for defining simulation platforms.
If \json{simulation} is set to \json{true} then this board is assumed to be a simulation platform.
This will make the \c{simulation_exit} function attempt to exit the simulator in case of catastrophic failures.

In addition, the \json{simulator} property can be the name of a program (or script) that can simulate images compiled for this board.
This will be run from the build directory and will be passed the absolute path of the firmware image when \command{xmake run} is used.
The build system will look for the simulator in the SDK directory and, failing that, in the path.
Exact paths can be provided by using \json{$\{sdk}} or \json{$\{board}} in the name of the simulator.
These will be expanded to the full path of the SDK or the directory containing the board description file, respectively.
