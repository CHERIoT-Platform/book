\chapter[numbering=false]{Preface}

The book that you are currently reading is not (yet) the first edition of the CHERIoT Programmers Guide.
It is a draft that contains most of the content that will go into the first edition, but has not yet completed copyediting.
As such, expect it to contain both grammatical and typographic errors.

This book is intended as a companion to the CHERIoT Platform.
The latest draft is published on the \href[src=https://cheriot.org/book]{CHERIoT web site} and, after the first edition is published, this will include any errata from the first edition.

\section{Acknowledgements}

A few months ago, this book was less than half its current length.
Completing this was supported by the UKRI Discribe Hub+, funded through the Economic and Social Research Council [ES/V003666/1].

The version that you are now reading has had some significant improvements in accuracy and structure thanks to some great feedback from technical reviewers Phil Day, Richard Edgar, Adam Finney, Hugo McNally.
Any remaining errors, omissions, or poor explanations are my responsibility.

The cats on the cover represent safe, secure, compartmentalisation (what is safer or more secure than a cat in a box?).
Each cat is in a separate, isolated, compartment, in the model for which CHERIoT was designed.

The cat photos were contributed by some wonderful people from the Fediverse.
Starting at the top left, numbered left to right then top to bottom, the photo credits are:

\begin{description}
	\item[tag="1, 3, 5, 10:"]{Photographer: James (@chongliss@mastodon.ie), Cats: Jiji (1), Luna (3, 5), and
		Felix (10).}
	\item[tag="2, 11:"]{Photographer: Cassian Lodge (@cassolotl@eldritch.cafe), Cat: Rosa.}
	\item[tag="4:"]{Photographer: Marin Benčević (@marinbenc@sigmoid.social)}
	\item[tag="6:"]{Photographer: Asta Halkjær From (@ahfrom@fedi.ahfrom.synology.me), Cat: Betty Rambo.}
	\item[tag="7:"]{Photographer: Victor Zverovich (@vitaut@mastodon.social), Cat: Luna (no relation).}
	\item[tag="8:"]{Photographer: Michael McWilliams (@MichaelMcWilliams@mas.to), Cat: Scotchy.}
	\item[tag="9:"]{Photographer: jarkman (@jarkman@chaos.social), Cat: Jack.}
	\item[tag="12:"]{Photographer: Isaac Freund (@ifreund@hachyderm.io), Cat: Marzipan.}
\end{description}

No generative AI was used in the creation of this image.
No artists work was appropriated without their consent.

\section{Reading and using examples}

Code listings in this book specify the file that they come from in the book's examples.
You can find the examples \href[src=https://github.com/CHERIoT-Platform/book/tree/main/text/examples]{in the book's git repository on GitHub}.

\begin{note}
	Before the first edition is finished, the examples will move to a separate repository, to make it easier to clone them and to include the dev container configuration.
\end{note}

Example code is pulled in from complete source files to ensure that everything that you see as a listing is valid code that will, at the very least, compile (and hopefully work).
You should be able to build and run all of the example code in the book yourself.

Listings have line numbers on the left.
These are the line numbers in the file, so you can read the extracted listings in context.

When you read the files that contain these listings, you will see comments like \c{// something#begin} and \c{// something#end}.
These are the markers for regions extracted and used in the book.

Syntax highlighting for this book is done by libclang (for C/C++) or TreeSitter (for Lua and Rego).
